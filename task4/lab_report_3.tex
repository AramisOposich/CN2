%%%%%%%%%%%%%%%%%%%%%%%%%%%%%%%%%%%%%%%%%
% CN2 Labreport template
%
% License:
% CC BY-NC-SA 3.0 (http://creativecommons.org/licenses/by-nc-sa/3.0/)
%
%%%%%%%%%%%%%%%%%%%%%%%%%%%%%%%%%%%%%%%%%

\documentclass[parskip=full]{scrartcl}

\usepackage{siunitx}  % Provides the \SI{}{} command for typesetting SI units
\usepackage{graphicx} % Required for the inclusion of images
\usepackage{booktabs} % nicer tables
\usepackage[noabbrev]{cleveref} % automatic references
\usepackage{listings} % typeset code
\usepackage[backend=biber]{biblatex}
\addbibresource{referenzen.bib}

\crefname{lstlisting}{listing}{listings} % for referencing code
\Crefname{lstlisting}{Listing}{Listings} % for referencing code

\usepackage[headsepline]{scrlayer-scrpage} % header
\ohead{Group 06} % right part of header
\ihead{Assignment 3} % left part of header

\lstset{basicstyle=\ttfamily} % monospaced font in listing



%----------------------------------------------------------------------------------------
%	DOCUMENT INFORMATION
%----------------------------------------------------------------------------------------

\begin{document}
\begin{titlepage}
    \centering
    \vspace*{2cm}
    {\Huge \textbf{Communication Networks 2}}\\
    SS 2021\\
    \vspace*{1cm}
    {\Large Assignment 4}
    \\\vspace*{3cm}
    {\Large \textbf{Group 06}}\\
    \vspace*{1cm}
    {\large 
        \begin{tabular}{l c c}
            Name & Mat.Nummer \\ \hline
            Paul Kloker & 12034928 \\
            Juan Aramis Oposich & 11701238
        \end{tabular}
    }
    \\\vspace*{7cm}
    \today
\end{titlepage}

%----------------------------------------------------------------------------------------
%	SECTION 1
%----------------------------------------------------------------------------------------
\section{Task description} \label{sec:task}
This task is based on the same network as the previous one and the goal is to determine its topology.
Thereby, an unknown host should be found and routing tables created for each one of the three routers.
Furthermore, the delay and packet loss of each link between the lab pc and every discovered host shall be measured, graphically represented and discussed.
%----------------------------------------------------------------------------------------
%	SECTION 2
%----------------------------------------------------------------------------------------
\section{Procedure} \label{sec:procedure}

\subsection{Host discovery with nmap} \label{subsec:nmap}
There are different techniques to discover active hosts on a network.
One of them is the use of nmap, which is a free and open source tool for network discovery and security auditing.
To find the missing host in \verb|10.0.0.0/16| the following command was used:
\begin{verbatim}
$ nmap --privileged -sn -n -T5 --min-parallelism 100 --min-hostgroup 100 
10.0.0.0/16
\end{verbatim}
To speed up the discovery process, which can take very long time in large networks, multiple options were added to the bare nmap command \verb|$ nmap 10.0.0.0/16|.
This reduced the waiting time to 23 minutes, which is still quite long. 

Table \ref{tab:nmap} shows the result of this nmap host discovery search in \verb|10.1.0.0/8| and \verb|10.0.0.0/8|.

\begin{table}[hb]
	\centering
	\begin{tabular}{|llll|}
		\hline
		\textbf{No.} & \textbf{Network} & \textbf{IP address} &\textbf{latency}  \\ 
		\hline
		1 & 10.0.0.0/8 & 10.0.4.1 &0.0075s\\
		%\hline
		2 & 10.0.0.0/8 & 10.0.4.2 &0.23s\\
		\hline
		3 & 10.0.0.0/8 & 10.0.120.1 &0.20s\\
		%\hline
		4 & 10.0.0.0/8 & 10.0.120.2 &0.0085s\\
		\hline
		5 & 10.0.0.0/8 & 10.0.132.1 &0.024s\\
		%\hline
		6 & 10.0.0.0/8 & 10.0.132.68 &0.18s\\
		\hline
		7 & 10.0.0.0/8 & 10.0.248.1 &0.78s\\
		%\hline
		8 & 10.0.0.0/8 & 10.0.248.2 &0.16s\\
		\hline
		\hline
		9 & 10.1.0.0/8 & 10.1.6.1 &0.18s\\
		%\hline
		10 & 10.1.0.0/8 & 10.1.6.110 &0.18s\\
		\hline
		11 & 10.1.0.0/8 & 10.1.7.1 &1.5s\\
		%\hline
		12 & 10.1.0.0/8 & 10.1.7.123 &0.78s\\
		\hline
	\end{tabular}
	\caption{Discovered IP addresses}
	\label{tab:nmap}
\end{table}

Later research and additional information showed that the 6th found IP address \verb|10.0.132.68| belongs to the missing host. 


\subsection{Ping measurements} \label{subsec:ping}
To identify which IP address belongs to the landline and satellite host a simple ping command was sent out to the according DNS names.
\verb|landline.cn2lab.cn.tuwien.ac.at| was resolved to \verb|10.1.6.110| and \verb|satellite.cn2lab.cn.tuwien.ac.at| to \verb|10.1.7.123|.

In order to obtain information about the network topology and the Round Trip Time (RTT) and loss rate of each host, the ping command was used as well.
For each IP address from \cref{tab:nmap} the following command was adapted and executed:
\begin{verbatim}
$ ping -c 50 -R 10.1.7.123 > 10_1_7_123.txt
\end{verbatim}
This delivered 50 individual measurements of the RRT which were then saved to a text file and are discussed in \cref{sec:data}.

With the \verb|-R| the record route option was activated.
That means all internet modules that route this message add their IP address to the IP option field.
This method is better than just using the command \verb|traceroute| because here the reverse path is recorded as well.

Some recorded routes show that the reverse path can be different from the forward path. 
This is for example the recorded route of the satellite host:
\begin{verbatim}
RR:  pc18.cn2lab.cn.tuwien.ac.at (192.168.88.118)
     10.0.120.2 (10.0.120.2)
     10.0.248.2 (10.0.248.2)
     10.1.7.1 (10.1.7.1)
     satellite.cn2lab.cn.tuwien.ac.at (10.1.7.123)
     satellite.cn2lab.cn.tuwien.ac.at (10.1.7.123)
     10.0.4.2 (10.0.4.2)
     border.cn2lab.cn.tuwien.ac.at (192.168.88.2)
     pc18.cn2lab.cn.tuwien.ac.at (192.168.88.118)
\end{verbatim}

\subsection{Network topology} \label{subsec:topo}
Using the data of the nmap and ping commands, the network topology could be identified and a network diagram created which can be seen in \cref{fig:topology}.
\begin{figure}[!ht]
	\centering % centering figure 
	\includegraphics[width=\textwidth]{images/topology.pdf} % importing figure
	\caption{Network diagram} 
	\label{fig:topology} % labeling to refer it inside the text
\end{figure}
Table \ref{tab:routing} shows the routing tables of the three routers. 
Some entries could not be identified by just using the ping command on the lab pc.

\begin{table}[hb]
	\centering

	\begin{tabular}{lll}
		\toprule
		\textbf{router} & \textbf{destination} & \textbf{via}  \\ \midrule
		r1 & 10.0.4.0/24 & 10.0.4.1 \\
		r1 & 10.0.120.0/24 & 10.0.120.2 \\
		r1 & 10.0.132.0/24 & 10.0.132.1 \\
		r1 & 10.0.248.0/24 & 10.0.120.1 \\
		r1 & 10.1.6.0/24 & 10.0.120.1 \\
		r1 & 10.1.7.0/24 & 10.0.120.1 \\
		r1 & 192.168.88.0/24 & 192.168.88.2\\
		\midrule
		r2 & 10.0.4.0/24 & - \\
		r2 & 10.0.120.0/24 & 10.0.120.1 \\
		r2 & 10.0.132.0/24 & - \\
		r2 & 10.0.248.0/24 & 10.0.248.2 \\
		r2 & 10.1.6.0/24 & 10.1.6.1 \\
		r2 & 10.1.7.0/24 & 10.0.248.1 \\
		r2 & 192.168.88.0/24 & 10.0.120.1\\
		\midrule
		r3 & 10.0.4.0/24 & 10.0.4.2 \\
		r3 & 10.0.120.0/24 & - \\
		r3 & 10.0.132.0/24 & - \\
		r3 & 10.0.248.0/24 & 10.0.248.1 \\
		r3 & 10.1.6.0/24 & - \\
		r3 & 10.1.7.0/24 & 10.0.7.1 \\
		r3 & 192.168.88.0/24 & 10.0.4.2\\
		\bottomrule
	\end{tabular}
	\caption{Routing table}
	\label{tab:routing}
\end{table}
\clearpage
\section{Data analysis and comparison} \label{sec:data}

One of the most important metrics of real-time communication is the end-to-end ($T_{EE}$) delay. In this context the ITU-T recommendation G.114 has classified the user acceptance for end-to-end delays in a VoIP call \cite{ITU-TRecommendationG.114}:

\begin{table}[hb]
	\centering
	\begin{tabular}{ll}
		\toprule
		\textbf{End-to-End delay} & \textbf{User experience} \\ \midrule
			$T_{EE} < 150$ & acceptable for all users \\
			$150 < T_{EE} < 300$ & noticeable quality degradation, but still acceptable for most users\\
			$T_{EE} \geq 300$ & not acceptable\\
			\bottomrule
		\end{tabular}
		\caption{Delay to user experience}
		\label{tab:delayEnd2End}
	\end{table}
	
One approach to determine the one-way delay is to timestamp the packet on the sender and check the time difference on the receiver. For our setup the round trip time could also be a approximately estimation  for the end-to-end delay. The captured delay is visualized in figure \ref{fig:one-way-delay}.

\begin{figure}[!ht]
	\centering % centering figure 
	\includegraphics[width=\textwidth]{images/oneWayDelay.pdf} % importing figure
	\caption{Propobility density of one way delay} 
	\label{fig:one-way-delay} % labeling to refer it inside the text
\end{figure}

The measurement contains two important aspects, one is the network probability distribution and the second is the overall delay. The landline connection seems to have a gaussian distribution and low latency, satellite on the other hand has an complete different shape and higher latency.

To further investigate the distribution it is important to understand the way how packets are treated in packet switched networks. As the topology section already explained the ping can propagate on different paths, this behavior could lead to different cost function on each link.

Another mechanism to utilize the network more efficiently is called load-balancing. A load-balancing router attempts to route Internet traffic optimally across two or more broadband connections to deliver a better experience to broadband users simultaneously accessing Internet applications.

An overview of the network statistics can be found in Table \ref{subsec:ping}. 

\begin{table}[hb]
	\centering
	\begin{tabular}{|lllrrrr|}
		\hline
		\textbf{Device} & \textbf{IP addresses} & \textbf{loss} &\textbf{RTT: min} & \textbf{avg} & \textbf{max }& \textbf{mdev}  \\ 
		\hline
		r1 	& 10.0.120.2 & 0 \% & 7.326 ms& 8.116 ms& 8.910 ms& 0.420 ms\\
		%\hline
		 	& 10.0.4.1 & 0 \% & 7.370 ms & 8.117 ms & 8.911 ms & 0.457 ms\\
		%\hline
		 	& 10.0.132.1 & 0 \% & 7.402 ms & 8.164 ms & 8.910 ms & 0.431 ms\\
		\hline
		r2	& 10.0.120.1 & 0 \% & 153.474 ms & 163.019 ms & 313.394 ms & 21.716 ms\\
		%\hline
			& 10.0.248.2 & 0 \% & 154.024 ms & 160.040 ms & 167.527 ms & 3.571 ms\\
		%\hline
			& 10.1.6.1 & 0 \% & 153.802 ms & 160.757 ms & 166.669 ms & 3.147 ms\\
		\hline
		r3	& 10.0.4.2 & 0 \% & 153.550 ms & 163.382 ms & 307.946 ms & 20.883 ms\\
		%\hline
			& 10.0.248.1 & 6 \% & 747.294 ms & 759.654 ms & 777.453 ms & 9.390 ms\\
		%\hline
			& 10.1.7.1 & 8 \% & 748.035 ms & 759.357 ms & 773.321 ms & 8.581 ms\\
		\hline
		Satellite & 10.1.7.123 & 6 \% & 745.616 ms & 759.713 ms & 776.987 ms & 10.471 ms \\
		\hline
		Landline & 10.1.6.110 & 0 \% & 154.818 ms & 159.911 ms & 165.216 ms & 3.349 ms \\
		\hline
		Unknown host & 10.0.132.68 & 0 \% & 7.323 ms & 8.189 ms & 8.966 ms & 0.518 ms\\
		\hline
	\end{tabular}
	\caption{Ping data from lab pc to different network devices (250 packets)}
	\label{tab:ping}
\end{table}


%----------------------------------------------------------------------------------------
%	SECTION 3
%----------------------------------------------------------------------------------------

\section{Conclusion}

As the second assignment showed the subjective quality and metrics like jitter are more acceptable on landline than on satellite call. This observation can also be seen on this paper. The measured packet loss and overall delay is much better on landline.

%----------------------------------------------------------------------------------------
%	SECTION X
%---------------------------------------------------------------------------------------

\printbibliography

\end{document}
