%%%%%%%%%%%%%%%%%%%%%%%%%%%%%%%%%%%%%%%%%
% CN2 Labreport template
%
% License:
% CC BY-NC-SA 3.0 (http://creativecommons.org/licenses/by-nc-sa/3.0/)
%
%%%%%%%%%%%%%%%%%%%%%%%%%%%%%%%%%%%%%%%%%

\documentclass[parskip=full]{scrartcl}

\usepackage{siunitx}  % Provides the \SI{}{} command for typesetting SI units
\usepackage{graphicx} % Required for the inclusion of images
\usepackage{booktabs} % nicer tables
\usepackage[noabbrev]{cleveref} % automatic references
\usepackage{listings} % typeset code
\usepackage[backend=biber]{biblatex}
\addbibresource{referenzen.bib}

\crefname{lstlisting}{listing}{listings} % for referencing code
\Crefname{lstlisting}{Listing}{Listings} % for referencing code

\usepackage[headsepline]{scrlayer-scrpage} % header
\ohead{Group 06} % right part of header
\ihead{Assignment 4} % left part of header

\lstset{basicstyle=\ttfamily} % monospaced font in listing



%----------------------------------------------------------------------------------------
%	DOCUMENT INFORMATION
%----------------------------------------------------------------------------------------

\begin{document}
\begin{titlepage}
    \centering
    \vspace*{2cm}
    {\Huge \textbf{Communication Networks 2}}\\
    SS 2021\\
    \vspace*{1cm}
    {\Large Assignment 4}
    \\\vspace*{3cm}
    {\Large \textbf{Group 06}}\\
    \vspace*{1cm}
    {\large 
        \begin{tabular}{l c c}
            Name & Mat.Nummer \\ \hline
            Paul Kloker & 12034928 \\
            Juan Aramis Oposich & 11701238
        \end{tabular}
    }
    \\\vspace*{7cm}
    \today
\end{titlepage}

%----------------------------------------------------------------------------------------
%	SECTION 1
%----------------------------------------------------------------------------------------
\section{Task description} \label{sec:task}

%----------------------------------------------------------------------------------------
%	SECTION 2
%----------------------------------------------------------------------------------------
\section{NS3 Model} \label{sec:procedure}


\begin{verbatim}
$ nmap --privileged -sn -n -T5 --min-parallelism 100 --min-hostgroup 100 
10.0.0.0/16
\end{verbatim}


\begin{table}[hb]
	\centering
	\begin{tabular}{|llll|}
		\hline
		\textbf{No.} & \textbf{Network} & \textbf{IP address} &\textbf{latency}  \\ 
		\hline
		1 & 10.0.0.0/8 & 10.0.4.1 &0.0075s\\
		%\hline
		2 & 10.0.0.0/8 & 10.0.4.2 &0.23s\\
		\hline
		3 & 10.0.0.0/8 & 10.0.120.1 &0.20s\\
		%\hline
		4 & 10.0.0.0/8 & 10.0.120.2 &0.0085s\\
		\hline
		5 & 10.0.0.0/8 & 10.0.132.1 &0.024s\\
		%\hline
		6 & 10.0.0.0/8 & 10.0.132.68 &0.18s\\
		\hline
		7 & 10.0.0.0/8 & 10.0.248.1 &0.78s\\
		%\hline
		8 & 10.0.0.0/8 & 10.0.248.2 &0.16s\\
		\hline
		\hline
		9 & 10.1.0.0/8 & 10.1.6.1 &0.18s\\
		%\hline
		10 & 10.1.0.0/8 & 10.1.6.110 &0.18s\\
		\hline
		11 & 10.1.0.0/8 & 10.1.7.1 &1.5s\\
		%\hline
		12 & 10.1.0.0/8 & 10.1.7.123 &0.78s\\
		\hline
	\end{tabular}
	\caption{Discovered IP addresses}
	\label{tab:nmap}
\end{table}
\clearpage
\section{Data analysis} \label{sec:data}
%Questions to be answered:
%1.) Why do the reported round-trip-times deviate from the sum of link delays? 
%2.) Why does the first ping reply have a higher round-trip-time than subsequent ones?

%3.) Which types of delay have been defined and used in Assignment 4?

%4.) How can we compute the transmission delay?
%5.) You simulated ping requests using a network consisting of one Point-to-Point link and one CSMA link. Which types of delay can the observed round-trip-delay be decomposed to?
%6.) What do we need ARP for?
%8.) Why is checksum computation turned off by default in a ns-3 simulation? Is it a problem/does it make a difference to omit checksums? !! The checksum is disabled by default to let the simulations run faster. !!

This chapter supports the understanding of the simulation model. Network model are been used to further estimate a real world scenario without a concrete build. Those models contains nodes and links just like a real internet system. For this problem we used NS-3 deploys a discrete-event simulator.

The assignment describes the needed network parameter like the network topology and the associated metrics. One task is to setup one point to point connection and also a CSMA link. those connections are implemented with \textbf{PointToPointHelper} and \textbf{CsmaHelper}. This classes are used to set up the IP addresses, MAC address and also channel delay. The channel delays, also propagation delays, is the time that it takes for a bit to reach from one end of a link to the other. The delay depends on the distance between the sender and the receiver, and the propagation speed of the wave signal.




%----------------------------------------------------------------------------------------
%	SECTION 3
%----------------------------------------------------------------------------------------

\section{Conclusion}



%----------------------------------------------------------------------------------------
%	SECTION X
%---------------------------------------------------------------------------------------

\printbibliography

\end{document}
