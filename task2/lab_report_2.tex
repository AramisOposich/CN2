%%%%%%%%%%%%%%%%%%%%%%%%%%%%%%%%%%%%%%%%%
% CN2 Labreport template
%
% License:
% CC BY-NC-SA 3.0 (http://creativecommons.org/licenses/by-nc-sa/3.0/)
%
%%%%%%%%%%%%%%%%%%%%%%%%%%%%%%%%%%%%%%%%%

\documentclass[parskip=full]{scrartcl}

\usepackage{siunitx}  % Provides the \SI{}{} command for typesetting SI units
\usepackage{graphicx} % Required for the inclusion of images
\usepackage{booktabs} % nicer tables
\usepackage[noabbrev]{cleveref} % automatic references
\usepackage{listings} % typeset code
\usepackage[backend=biber]{biblatex}
\addbibresource{referenzen.bib}

\crefname{lstlisting}{listing}{listings} % for referencing code
\Crefname{lstlisting}{Listing}{Listings} % for referencing code

\usepackage[headsepline]{scrlayer-scrpage} % header
\ohead{Group 06} % right part of header
\ihead{Assignment 1} % left part of header

\lstset{basicstyle=\ttfamily} % monospaced font in listing



%----------------------------------------------------------------------------------------
%	DOCUMENT INFORMATION
%----------------------------------------------------------------------------------------

\begin{document}
\begin{titlepage}
    \centering
    \vspace*{2cm}
    {\Huge \textbf{Communication Networks 2}}\\
    SS 2021\\
    \vspace*{1cm}
    {\Large Assignment 2}
    \\\vspace*{3cm}
    {\Large \textbf{Group 06}}\\
    \vspace*{1cm}
    {\large 
        \begin{tabular}{l c c}
            Name & Mat.Nummer \\ \hline
            Paul Kloker & 12034928 \\
            Juan Aramis Oposich & 11701238
        \end{tabular}
    }
    \\\vspace*{7cm}
    \today
\end{titlepage}

%----------------------------------------------------------------------------------------
%	SECTION 1
%----------------------------------------------------------------------------------------
\section{Task and Protocol Description} \label{sec:task}
Aufgaben, kurz SIP und SDP, RTP, kurz codecs, (kurz MOS und ACR/DCR)
%----------------------------------------------------------------------------------------
%	SECTION 2
%----------------------------------------------------------------------------------------
\section{Procedure} \label{sec:procedure}

<<<<<<< HEAD
\subsection{Linphone setup and SIP registration} \label{subsec:setup}
- welche zugangsdaten wo angeben... 
=======
Hallo
>>>>>>> 87e11a58333fe4564013f696e0e012baa5c473de

- Signaling (Sequenzdiagramm) des Registrierungsvorgangs + Secret message 
\subsection{Capturing process and SIP signaling} \label{subsec:capture}
- Wie sind wir vorgegangen um alle unterschiedlichen Codecs zu capturen... (+cn2\_sbs\_capture  tool)


- SIP Signaling mit Sequenzdiagramm beschreiben "verify that both, signaling and media work correctly."

- SDP genauer anschauen (stehen die ausgewählten codecs drinnen...)
\subsection{Audio codec quality comparison} \label{subsec:audio}
- Audio codecs subjektiv  vergleichen (vielleicht nach der Mean Opinion Score)

- RTP Analyse + Grafen
\subsection{Video codec quality comparison} \label{subsec:video}
- Video codecs subjektiv vergleichen  

z.B.\verb| https://en.wikipedia.org/wiki/Subjective_video_quality | 
  
Das ist so ähnlich wie MOS:
https://www.irisa.fr/armor/lesmembres/Mohamed/Thesis/node146.html

- RTP Analyse + Grafen
%----------------------------------------------------------------------------------------
%	SECTION 3
%----------------------------------------------------------------------------------------
\section{Conclusion}

Was sind die besten Codecs für welche Situation. 
%----------------------------------------------------------------------------------------
%	SECTION X
%---------------------------------------------------------------------------------------

\printbibliography

\end{document}
